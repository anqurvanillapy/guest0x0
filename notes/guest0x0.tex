\section{Introduction}
Throughout this tutorial, \fbox{boxes} will be used in the following two cases:
\begin{itemize}
\item to clarify the precedences of symbols when formulae become too large. \\
e.g. \fbox{$\Gvdash \fbox{$\lam x M$}~:~\fbox{$(y:A)\to B$}
\Leftarrow \fbox{$\lam x u$}$}.
\item to distinguish type theory terms from natural language text. \\
e.g. ``we combine a term \fbox{$a$} with a term \fbox{$b$} to get a term \fbox{$a~b$}''.
\end{itemize}
\subsection{Target Audience}
This tutorial assumes familiarity with the following:
\begin{itemize}
\item Dependent type theory concepts, such as formation rules,
introduction rules, eliminators, etc., and functional programming.
\item Programming and theorem proving in a proof assistant based on dependent type theories.
\item The ability to translate (simple) typing rules into a type-checking procedures
and combine them into an algorithm.
\item Basic understanding of De Morgan cubical type theory~\cite{CCHM,CHM}
(hereafter as \CTT{}), including the interval type $\II$, the path type,
and the idea of representing $n$-dimensional cubes using terms with interval variables in it.
\end{itemize}
This tutorial will not treat substitution formally --
variable names are assumed to respect capture-avoiding substitution.
In the implementations, any binding representation that works for untyped
$\lambda$-calculus should work for the type theory introduced in this tutorial.
\begin{notation}\label{not:pre}
We will prefer using $x, y, z$ for variables
and other Latin letters like $u,v,a,b,c,A,B,C$ for terms
(preferably uppercase for types and lowercase for terms).

Instead of the more traditional \fbox{$\lambda x.b$},
the notation for $\lambda$-abstraction is \fbox{$\lam x b$} 
following the style of the Arend language.
We will also use the conventional shorthand
\fbox{$(x:A)~(y:B)\to C$} for nested $\Pi$-types.

Substitution is denoted by $u[v/x]$.
One may think of this notation as ``fractional multiplication'' $u\times \frac v x$,
where the denominator $x$ is cancelled out from $u$ and \textit{replaced} with the numerator $v$.
Other authors may use $u[x\mapsto v]$, $u[x:=v]$, $[v/x]u$, etc.

Definitional equality (a.k.a. judgmental equality) is denoted $u\equiv v$.
\end{notation}
\begin{notation}\label{not:syntax-def}
We will write \fbox{$u, A::=$} for syntax definition of terms,
and define the syntax of \CTT{} by extending the syntax of Martin-L\"of type theory
with a few term former at a time, instaed of putting everything together
in a single, unified BNF grammar.
The typing rules will be introduced similarly.

We will extend the BNF grammar with the \textit{list} operator like \fbox{$\overline{(x_i:A_i)}\to B$},
which means that the string below the line can be repeated one or more times,
optionally indexed by a subscript such as $i$.
\end{notation}
Some quick warm-ups:
\begin{exercise}
Translate the following typing rules into an algorithmic description:
\begin{mathpar}
\inferrule{\Gamma,x:A\vdash b:B[x/y]}{\Gvdash \fbox{$\lam x b$} : (y:A)\to B} \and
\inferrule{\Gvdash u : (x:A)\to B \\ \Gvdash v:A}{\Gvdash u~v:B[v/x]}
\end{mathpar}
Which of them is an introduction rule, and which is an elimination rule?
\end{exercise}
\begin{exercise}\label{ex:concat-sym}
Consider path concatenation and symmetry in \CTT{}:
\begin{align*}
\concat&:(p:a=b)\to(q:b=c)\to{a=c}\\
\sym&:(p:a=b)\to{b=a}
\end{align*}
Define both using \hcomp{} on the following squares,
preferably in a cubical programming language:
% https://q.uiver.app/?q=WzAsOCxbMCwwLCJhIl0sWzEsMCwiYyJdLFswLDEsImEiXSxbMSwxLCJiIl0sWzIsMSwiYSJdLFsyLDAsImIiXSxbMywxLCJhIl0sWzMsMCwiYSJdLFswLDEsIlxcY29uY2F0KHAsIHEpIiwwLHsic3R5bGUiOnsiYm9keSI6eyJuYW1lIjoiZGFzaGVkIn19fV0sWzIsMCwiIiwwLHsibGV2ZWwiOjIsInN0eWxlIjp7ImhlYWQiOnsibmFtZSI6Im5vbmUifX19XSxbMiwzLCJwIiwyXSxbMywxLCJxIiwyXSxbNCw1LCJwIl0sWzQsNiwiIiwyLHsibGV2ZWwiOjIsInN0eWxlIjp7ImhlYWQiOnsibmFtZSI6Im5vbmUifX19XSxbNiw3LCIiLDIseyJsZXZlbCI6Miwic3R5bGUiOnsiaGVhZCI6eyJuYW1lIjoibm9uZSJ9fX1dLFs1LDcsIlxcc3ltKHApIiwwLHsic3R5bGUiOnsiYm9keSI6eyJuYW1lIjoiZGFzaGVkIn19fV1d
\[\begin{tikzcd}
	a & c & b & a \\
	a & b & a & a
	\arrow["{\concat(p, q)}", dashed, from=1-1, to=1-2]
	\arrow[Rightarrow, no head, from=2-1, to=1-1]
	\arrow["p"', from=2-1, to=2-2]
	\arrow["q"', from=2-2, to=1-2]
	\arrow["p", from=2-3, to=1-3]
	\arrow[Rightarrow, no head, from=2-3, to=2-4]
	\arrow[Rightarrow, no head, from=2-4, to=1-4]
	\arrow["{\sym(p)}", dashed, from=1-3, to=1-4]
\end{tikzcd}\]
\end{exercise}
\subsection{Motivation}
This tutorial is intended to help readers get more familiar with how
\CTT{} works, what difficulties it is having in implementations,
what it can already do, and what it cannot do yet.

\CTT{} is a type theory evolved from a model using Kan cubical sets~\cite{CubicalSets},
which uses sophisticated mathematics. Computer scientists, on the other hand,
usually do not have relevant courses taught in their undergraduate program.
However, it is also the computer scientists who are suppose to implement \CTT{} as
programming languages. This tutorial tries to help those who did not study homotopy
theory, but wish to learn the internals of \CTT.

\CTT{} extends Martin-L\"{o}f type theory with a huge amount of new constructions,
especially the typing rules are written in a very compact way (like in~\cite{HCompPDF}).
This tutorial aims to discuss them from an algorithmic perspective,
and hopefully to inspire more people to implement cubical type theory,
to fuse these ideas into other work, or just to worship these brilliant ideas.

This tutorial is a by-product of an experiment in implementing \CTT, called \GuestName,
a project created to encourage a particular person to learn \CTT.
The story ends up in the worst way: the person did not learn \CTT, and instead created
a new project to encourage the author of \GuestName{} to learn extensional type theory.

\section{Type checking cubes}\label{sec:tyck-cube}
This section motivates and explains typing judgments with cofibrations in the context.

\subsection{The interval $\II$ and contexts}\label{sub:interval}
In \CTT, we have the interval type:
\[\vdash \isType\II\quad \vdash \lcon:\II \quad \vdash \rcon:\II\]
The interval type and its products are used to represent dimensions
(ignoring the De Morgan structures for now).
\begin{example}\label{ex:interval-in-ctx}
Suppose \fbox{$x:\II\vdash\isType{A}$} and \fbox{$x:\II\vdash u:A$}.
From a semantical or a topological perspective, we can say:
\begin{itemize}
\item $A$ is a (type) line between $A[\lcon/x]$ and $A[\rcon/x]$.
\item $u$ is a (term) line between $u[\lcon/x]$ and $u[\rcon/x]$.
\item The type of a line is a line, so the type of $u$ is $A$.
\end{itemize}
Then, because \emph{typing relations are preserved by substitution},
the following typing relations hold:
\begin{mathpar}
\inferrule{}{u[\rcon/x]:A[\rcon/x]}\and
\inferrule{}{u[\lcon/x]:A[\lcon/x]}
\end{mathpar}
We may visualize the fact as:
% https://q.uiver.app/?q=WzAsNixbMCwwLCJ1W1xcbGNvbi9pXSJdLFswLDEsInVbXFxyY29uL2ldIl0sWzIsMCwiQVtcXGxjb24vaV0iXSxbMiwxLCJBW1xcbGNvbi9pXSJdLFsxLDBdLFsxLDFdLFswLDEsInUiXSxbMiwzLCJBIl0sWzQsNSwiOiIsMSx7InN0eWxlIjp7ImJvZHkiOnsibmFtZSI6Im5vbmUifSwiaGVhZCI6eyJuYW1lIjoibm9uZSJ9fX1dXQ==
\[\begin{tikzcd}
	{u[\lcon/x]} & {} & {A[\lcon/x]} \\
	{u[\rcon/x]} & {} & {A[\lcon/x]}
	\arrow["u", from=1-1, to=2-1]
	\arrow["A", from=1-3, to=2-3]
	\arrow["{:}"{description}, draw=none, from=1-2, to=2-2]
\end{tikzcd}\]
\end{example}
From~\cref{ex:interval-in-ctx} we motivate the following notational
convention for contexts in \CTT, as in~\cref{not:ctx}.
\begin{notation}\label{not:ctx}
Typing judgments are written as \fbox{$\PGvdash \isType{A}$} and \fbox{$\PGvdash u:A$},
where \fbox{$\Phi;\Gamma$} is the usual \textit{context} in type theories,
with variables classified into two groups: if a variable has type $\II$,
it goes to $\Phi$, otherwise it goes to $\Gamma$.
This convention is borrowed from~\cite{ABCFHL}.
\end{notation}
Note that~\cref{not:ctx} does not imply that contexts has to be classified
in the implementations. The \GuestName{} type checker mix intervals
and other bindings in a unified context, just like usual dependent type checkers.
\begin{remark}\label{rem:line-ori}
Consider \fbox{$x:\II\vdash u:A$} and \fbox{$\vdash v:\II\to A$}.
Usually both are referred to as a \textit{line}, but they are very different.
Suppose we weaken the context with $y:\II$ to be a $2$-dimensional space,
in which $u$ exists as a line:
\carloCTikZ{\carloXy
\node (0) at (0 , 0) {\textbullet} ;
\node (1) at (1 , 0) {\textbullet} ;
\draw[->] (0) -- (1) node [midway, above] {$u$};}
Note that the orientation of $u$ is fixed to be horizontal.
However, for $v$, we can apply either $x$ or $y$ to get a line oriented differently:
\carloCTikZ{\carloXy
\node (0) at (0 , 0) {\textbullet} ;
\node (1) at (1 , 0) {\textbullet} ;
\node (2) at (0 , 1) {\textbullet} ;
\draw[->] (0) -- (1) node [midway, above] {$v~x$};
\draw[->] (0) -- (2) node [midway, left] {$v~y$};}
So, interval application may also be thought of as
\textit{placing an $n$-dimensional cube at the given orientation}.
\end{remark}
\begin{remark}\label{rem:sq-ori}
Unlike lines as discussed in~\cref{rem:line-ori}, squares are much more flexible.
Consider \fbox{$\vdash u:\II\to\II\to A$} in a $2$-dimensional context,
there are already two different ways to place it:
\carloCTikZ{\carloXy
\carloSqBullets
\fill [pattern color=gray,pattern=horizontal lines] (0,0) rectangle (1,1) ;
\node (c) at (0.5, 0.5) {$u~x~y$} ;
\shiftTikZ{1.5,0}{
\carloSqBullets
\fill [pattern color=gray,pattern=vertical lines] (0,0) rectangle (1,1) ;
\node[rotate=90,xscale=-1] (c) at (0.5, 0.5) {$u~y~x$} ;
}}
Note that these two placements are symmetric with respect to the diagonal.

In case of contexts and cubes of higher dimensions,
the situations are much more complicated.
For example, with one more dimension $z:\II$,
we may place $u$ in three orientations:
\carloCTikZ{\carloXyz
\refcube{TopUXY}
\shiftTikZ{0,0.3}{\refcube{FrontUXY}}
\shiftTikZ{1.3,0.2}{\refcube{LeftUXY}}
}
Note that all of them can also be reflected by their diagonals.
\end{remark}

\subsection{Partial elements}\label{sub:partial}
In \CTT{}, the idea that \textit{open shapes can be filled}
is the core concept that makes terms in type theory space-like,
and to do so the \textit{Kan operation} are added to \CTT{} as a structure.

The motivation is that with only the interval type and Martin-L\"of type theory,
we may not be able to describe every cube that makes geometric or topological sense.
For example, the squares in~\cref{ex:concat-sym} make perfect sense in
geometry or topology, but in \CTT{}, they have to be constructed using the Kan operation.
Kan operation takes a description of some parts of an $n$-dimensional cube
(an \textit{incomplete} cube, or a \textit{partial} cube) and completes it.
To describe the input of Kan operations, we introduce partial elements.
\begin{terminology}
We follow the terminology in~\cite{CubicalAgda}.
In~\cite{CCHM}, partial elements are called \textit{systems}.
\end{terminology}
When we write \fbox{$x:\II,y:\II\vdash u:A$},
we are describing the following 2-dimensional cube, which is a square:
\carloCTikZ{\carloXy
\carloSqBullets
\fill [pattern color=gray,pattern=north west lines] (0,0) rectangle (1,1) ;
\node (center) at (0.5, 0.5) {$u:A$} ;}
For simplicity we assume \fbox{$\vdash \isType{A}$}, say, $A$ does not depend on $x$ or $y$.

Suppose we want to use the Kan operation to create
a square \fbox{$x:\II,y:\II\vdash u:A$}
such that its top-left corner is a point $a:A$, and the line on its
right-hand side is a line $y:\II \vdash v:A$:
\carloCTikZ{\carloXy
\foreach \y in {0,1} \node (1\y) at (1 , \y) {\textbullet} ;
\node (00) at (0, 0) {$a$};
\node (01) at (0, 1) {\textbullet};
\draw (10) -- node [right] {$v$} (11); %
% \fill [pattern color=gray,pattern=north west lines] (0,0) rectangle (1,1) ;
% \node (center) at (0.5, 0.5) {$u:A$} ;
}

Translating that into type theory,
the goal is to construct a term $u$ such that the following holds:
\begin{align*}
u[\lcon/x, \rcon/y]&\equiv a\\
u[\rcon/x]&\equiv v
\end{align*}
The construction will be discussed in later sections,
and for now we focus on how to describe these partial boundaries
(also known as \textit{configurations} of a cube).
We introduce a straightforward syntax called \textit{partial elements}
for these cubes, e.g. the above partial element is written as:
\[\LRbbar{\begin{array}{rc}
  x=\lcon \land y=\rcon&\mapsto a\\
  x=\rcon &\mapsto v
\end{array}}\]
To define this formally, we need to define the syntax of the left-hand-side of $\mapsto$.
They are called \textit{cofibrations} in \CTT.
\begin{terminology}
In~\cite{CCHM}, cofibrations are called \textit{face restrictions},
which is more (geometrically) intuitive but also longer.
\end{terminology}
The syntax of a cofibration is defined to be a disjunction normal form
(a disjunction list of conjunctions) of face \textit{conditions}
like \fbox{$x=\lcon$} or \fbox{$x=\rcon$}, as in~\cref{fig:cofib} (recall~\cref{not:syntax-def}).
\begin{figure}[h!]
\[\begin{array}{rll}
  \cond ::= & x=\lcon \mid x=\rcon & \text{condition} \\
  \conj ::= & \cond~\overline{\land~\cond} & \text{conjunction} \\
  \disj ::= & \bot \mid \top \mid \conj~\overline{\lor~\conj} & \text{disjunction}
\end{array}\]
\caption{Syntax of cofibrations}\label{fig:cofib}
\end{figure}

The typing rules for cofibrations is also straightforward, as in~\cref{fig:tyck-cofib}.
\begin{figure}[h!]
\begin{mathpar}
\inferrule{(x:\II) \in \Phi}{\Phi\vdash \isCond{x=\lcon}} \and
\inferrule{(x:\II) \in \Phi}{\Phi\vdash \isCond{x=\rcon}}\and
\inferrule{\forall i. (\Phi\vdash \isCond{\cond_i})}{\Phi\vdash \bigwedge\nolimits_{i} \cond_i}\and
\inferrule{\forall i. (\Phi\vdash \conj_i)}{\Phi\vdash \bigvee\nolimits_{i} \conj_i}\and
\inferrule{}{\Phi\vdash \top} \and
\inferrule{}{\Phi\vdash \bot}
\end{mathpar}
\caption{Typing rules of cofibrations}\label{fig:tyck-cofib}
\end{figure}

The meaning of cofibrations is simple. Suppose we are in a $3$-dimensional context,
which means there are $3$ intervals in the context, i.e. \fbox{$\Phi:=x:\II,y:\II,z:\II$}.
Then:
\begin{itemize}
\item As mentioned before, the term \fbox{$\Phi\vdash u:A$} corresponds to
(the \textit{filling} of) a $3$-dimensional cube:
\carloCTikZ{\carloXyz \refcube{Empty}}
\item A $\cond$ specifies a $2$-dimensional cube (a square) face in $\Phi$, e.g. \fbox{$x=\lcon$}
corresponds to the following square:
\carloCTikZ{\carloXyz \refcube{XEquivL}}
\item A $\conj$ specifies any $n$-cube (for $n\leq 2$) in $\Phi$, e.g.
\fbox{$x=\lcon\land y=\rcon$} specifies a line $v$,
and \fbox{$x=\lcon\land y=\rcon\land z=\lcon$} specifies a point $\star$:
\carloCTikZ{\carloXyz \refcube{XYLR}}
\item A $\disj$ talks about several $n$-cubes (for $n\leq 2$) in $\Phi$ at the same time, e.g.
\fbox{$x=\lcon\lor x=\rcon$} corresponds to the following two squares:
\carloCTikZ{\carloXyz \refcube{LRFaces}}
\item A $\bot$ cofibration is called the \textit{absurd} cofibration,
which specifies nothing.
\item A $\top$ cofibration is called the \textit{truth} cofibration,
which specifies everything.
\end{itemize}
Using cofibrations, we define the syntax of partial elements
by extending the syntax for terms in~\cref{fig:parEl}:
\begin{figure}[h!]
\[\begin{array}{rl}
    \face::=&\overline{\conj\mapsto u}\\
    u,A::=&\lrbbar{\face} \mid \cdots~\text{(other term formers, recall~\cref{not:syntax-def})}
\end{array}\]
\caption{Syntax of partial elements}\label{fig:parEl}
\end{figure}

Note that we also need to define the type of partial elements
(hereafter as \textit{partial types}),
and the type needs to contain the following two pieces of information:
\begin{itemize}
\item The faces being specified, $\disj$.
\item The type of the faces, $A$.
\end{itemize}
Thus we directly define the syntax of partial types as in~\cref{fig:parTy}:
\begin{figure}[h!]
\[u,A::= \Partial{\disj}{A} \mid \cdots~\text{(see~\cref{fig:parEl})}\]
\caption{Syntax of partial types}\label{fig:parTy}
\end{figure}

There are several advantages to arrange the cofibrations
as disjunction-normal forms:
\begin{itemize}
\item We put $\conj$ to the left-hand-side of $\mapsto$,
so every clause in a partial element specifies a single face.
\item It is easy to get the type of a partial element:
since each clause specifies a face using $\conj$,
their disjunction is the $\disj$ in the corresponding partial type.
\end{itemize}
It remains to derive the evaluation and typing rules for partial elements.

\subsection{Reducing partial elements by cofibrations}\label{sub:red-cofib}
We reduce well-typed (we will define well-typedness of
partial elements later in~\cref{sub:tyck-cofib}) partial elements by
iterating their face clauses. Consider the following face clause
with no conjunction:
\[x=\lcon\mapsto u\]
We may substitute the variable $x$ with three possible terms of type $\II$:
\begin{itemize}
\item Another variable $y$. In this case, we simply replace $x$ with $y$
and the face becomes $y=\lcon\mapsto u$.
\item $\lcon$, i.e. take the face that $x=\lcon$ in this partial element.
Then, evidently, we get the face $u$, so the partial element should reduce to $u$.
Other faces can be safely dropped.
In this case, we say that the face is \textit{satisfied}.
\item $\rcon$, i.e. take the face that $x=\rcon$,
which is unspecified by this face, so we drop this particular face and
proceed with the rest of the faces.
In this case, we say that the face is \textit{contradicted}.
\end{itemize}
With the presence of conjunctions, we iterate through each $\cond$,
drop the conditions that are satisfied, and drop the faces
if one of their conditions is contradicted.

There are two more special cases in a conjunction $\conj$:
\begin{itemize}
\item It can be self-contradictory.
This happens when for a variable $x$, we have both $x=\lcon\in\conj$
and $x=\rcon\in\conj$. In this case, we also remove the face.
\item It can contain duplicated information.
For example, we may write $x=\lcon\land x=\lcon$.
It is encouraged to deduplicate these conditions for spatial efficiency.
In this tutorial, we assume deduplication of cofibrations everywhere.
\end{itemize}
By that we may claim the following:
\begin{prop}\label{prop:unique}
In a $\conj$ cofibration, every variable appears uniquely in a $\cond$.
\end{prop}

Substitution may also change partial types \fbox{$\Partial\disj A$}.
The rules are essentially the same as partial elements,
but in case of a face is satisifed, we reduce
the partial type into the annotated type $A$.

\subsection{Type checking under cofibrations}\label{sub:tyck-cofib}
We start from an example of type checking a partial element.
\begin{example}\label{ex:parTyck}
Consider $\Phi:=x:\II, y:\II,z:\II$ and the following:
\begin{mathpar}
\inferrule{}{\Gvdash\isType{A}} \and
\inferrule{}{\Gvdash u:\II\to\II\to A}\and
\inferrule{}{\Gvdash v:\II\to\II\to A}
\end{mathpar}
We want to construct the following partial element (recall~\cref{rem:sq-ori}):
\carloCTikZ{\carloXyz \refcube{BBFaces}}
Recall~\cref{fig:parEl}, we may directly translate that into:
\[\LRbbar{\begin{array}{rll}
z&=\rcon&\mapsto u~x~y\\
y&=\lcon&\mapsto v~x~z
\end{array}}\]
Note that they share the same line \fbox{$z=\rcon\land y=\lcon$}
(obtained by taking the conjunction of both cofibrations).
This line is represented as a cofibration, and we can also use it as
a \textit{substitution}, i.e. $[\rcon/z,\lcon/y]$.
Substituting with this line corresponds to the operation of
taking this line from a cube (that has this line).
Since both faces have this line, we may represent the shared line
by substituting either of them:
\begin{align*}
(u~x~y)[\rcon/z,\lcon/y]&\implies u~x~\lcon\\
(v~x~z)[\rcon/z,\lcon/y]&\implies v~x~\rcon
\end{align*}
However, these two lines are actually the same line,
so they have to be definitionally the same, i.e. $u~x~\lcon\equiv v~x~\rcon$.
In this case, we say that the given two faces \textit{agree}.
For a well-defined partial element, every pair of faces should agree.
The typing rule ends up like this:
\begin{mathpar}
\inferrule{\Hint{\Gvdash\isType{A}}\\
\PGvdash u~x~y:A\\
\PGvdash v~x~z:A\\\\
\PGvdash u~x~\lcon\equiv v~x~\rcon:A}{\PGvdash\LRbbar{\begin{array}{rll}
z&=\rcon&\mapsto u~x~y\\
y&=\lcon&\mapsto v~x~z
\end{array}}:\Partial{(z=\rcon\lor y=\lcon)}A}
\end{mathpar}
\end{example}

We generalize~\cref{ex:parTyck} to an arbitrary partial element (let $i\in I$ for some index set $I$):
\[\lrbbar{\overline{\conj_i\mapsto u_i}}\]
We need to find an algorithm that makes sure every pair of faces agree.
For two faces $\conj_i\mapsto u_i$ and $\conj_j\mapsto u_j$,
from~\cref{ex:parTyck} we know that they overlap at $\conj_i\land\conj_j$
(in case they do not overlap, this cofibration is self-contradictory).
The rule becomes something like:
\begin{mathpar}
\inferrule{\Hint{\Gvdash\isType{A}}\\
\forall i\in I.(\Phi\vdash \conj_i)\\
\forall i\in I.(\PGvdash u_i:A)\\\\
\forall i,j\in I.(\PGvdash[\text{overlap-check}])}
{\PGvdash\lrbbar{\overline{\conj_i\mapsto u_i}}
:\Partial{\left(\bigvee\nolimits_{i\in I} \conj_i\right)}A}
\end{mathpar}
In~\cref{ex:parTyck}, the overlap-check step is a conversion check
between a \textit{substituted version} of the two faces $u_i$ and $u_j$.
For convenience, we introduce the typing judgment as in~\cref{fig:conv-cofib}.
\begin{figure}[h!]
\[\Phi;\Gamma,\conj\vdash u\equiv v:A\]
\caption{Conversion check under a conjunction cofibration}\label{fig:conv-cofib}
\end{figure}

To implement this judgment, we convert $\conj$ into a substitution
(this is possible due to~\cref{prop:unique}),
apply to the three terms on the right-hand side of $\vdash$,
and then apply the normal conversion check under $\Phi;\Gamma$.
The idea in~\cref{fig:conv-cofib} can be extended in the following ways:
\begin{itemize}
\item To type check under several conjunctions,
we combine them together using conjunctions.
\item To do a normal type-check under conjunctions,
e.g. \fbox{$\Phi;\Gamma,\conj\vdash u:A$}, we apply the substitution to $A$
and run the normal type-check.
\item To check anything that results in a \textit{yes} or \textit{no},
we may check them under a disjunction cofibration $\disj$
by iterating the conjunctions in $\disj$ and run the check under that conjunction.
We return \textit{yes} only when all these checks return \textit{yes},
and return \textit{no} otherwise.
We may further extend the judgment to allow multiple disjunctions too.
\end{itemize}
These ideas give rise to the judgments in~\cref{fig:tyck-cofib}.
\begin{figure}[h!]
\begin{align*}
\Phi;\Gamma,\overline{\conj_i}&\vdash u\equiv v:A\\
\Phi;\Gamma,\overline{\disj_i}&\vdash u\equiv v:A\\
\Phi;\Gamma,\overline{\conj_i}&\vdash u:A\\
\Phi;\Gamma,\overline{\conj_i}&\vdash \isType{A}
\end{align*}
\caption{Judgments under cofibrations}\label{fig:tyck-cofib}
\end{figure}

With all these preparations we can get the typing rule
for partial elements (note that we also add support for $A$ to
refer to variables in $\Phi$), as in~\cref{fig:tyck-par}.
\begin{figure}[h!]
\begin{mathpar}
\inferrule{\Phi\vdash \disj\\ \PGvdash\isType{A}}{\PGvdash \isTypeBox{\Partial{\disj}A}}\and
\inferrule{
\forall i\in I.(\Phi\vdash \conj_i)\\
\forall i\in I.(\Phi;\Gamma,\conj_i\vdash u_i:A)\\\\
\forall i,j\in I.(\Phi;\Gamma,\conj_i,\conj_j\vdash u_i\equiv u_j:A)}
{\PGvdash\lrbbar{\overline{\conj_i\mapsto u_i}}
:\Partial{\left(\bigvee\nolimits_{i\in I} \conj_i\right)}A}
\end{mathpar}
\caption{Typing rule of partial elements}\label{fig:tyck-par}
\end{figure}

\subsection{Generalized path type}\label{sub:extTy}
In~\cite{CCHM}, the path type is defined using the following rules
(slightly paraphrased for notational consistency):
\begin{mathpar}
\inferrule{\Phi,x:\II;\isType A\\ \PGvdash a:A[\lcon/x]\\ \PGvdash b:A[\rcon/x]}
{\PGvdash\isTypeBox{\PathTy{\lam x A} a b}}\and
\inferrule{\Phi,x:\II;\Gvdash u:A}
{\PGvdash \plam x u : \PathTy{\lam x A}{(u[\lcon/x])}{(u[\rcon/x])}}\and
\inferrule{\Hint{\Phi,x:\II;\isType A}\\\Phi,x:\II;\Gvdash u:A\\\PGvdash v:\II}{
\PGvdash \papp{(\plam x u)}v \equiv u[v/x]:A[v/x]}\and
\inferrule{\Hint{\Phi,x:\II;\isType A}\\\PGvdash u:\PathTy{\lam x A} a b}{
\PGvdash \papp u\lcon \equiv a:A[\lcon/x] \\ \text{and} \\ \PGvdash \papp u\rcon \equiv b:A[\rcon/x]}
\end{mathpar}
We may rephrase these rules using the idea of cofibrations and partial elements:
\begin{itemize}
\item The path type \fbox{$\PathTy{\lam x A} a b$}
is a $\Pi$-type \fbox{$(x:\II)\to A$} carrying a partial element, denoted as:
\[\ExtTy{x}{A}{rll}
{x&=\lcon&\mapsto a\\
x&=\rcon&\mapsto b}\]
\item The introduction rule takes a type
\fbox{$\extTy{x}{A}{\overline{\conj_i\mapsto v_i}}$} and check the following:
\begin{itemize}
\item \fbox{$\Phi,x:\II;\Gvdash u:A$}, just like checking \fbox{$\lam x u:(x:\II)\to A$}.
\item \fbox{$\forall i. (\Phi,x:\II;\Gamma,\conj_i\vdash u\equiv v_i:A)$},
like saying ``$u$ \textit{matches} every given face''.
\end{itemize}
\item The elimination rule is the same as $\Pi$-types.
\item The computation rules combine the computation rule of $\Pi$-types
and the reduction of partial elements (defined in~\cref{sub:red-cofib}).
\end{itemize}
In particular, we can relax the formation rule to allow multiple intervals
and an arbitrary partial element (but the cofibrations may only use the
provided intervals), and the rest of the rules will still work.
This gives us a better definition of the path type,
which is more convenient to work with.
We may specify only one endpoint (unlike a traditional path,
which always requires two endpoints) of a generalized path as in~\cref{ex:gconcat},
or specify the inner boundaries of squares as in~\cref{ex:gsquare}.

We extend the syntax with generalized paths in~\cref{fig:path}.
\begin{figure}[h!]
\[u,A::= \extTy{\overline{x}}{A}{\overline{\face}}
\mid \plam x u \mid \papp u v
\mid \cdots~\text{(see~\cref{fig:parTy})}\]
\caption{Syntax of generalized paths}\label{fig:path}
\end{figure}

The typing rules are defined in~\cref{fig:tyck-path}.
Note that we make use of reduction on partial elements (\cref{sub:red-cofib}).
\begin{figure}[h!]
\begin{mathpar}
\inferrule{\Phi\overline{,x:\II};\Gvdash \isType{A}\\
\forall i.(\overline{x:\II}\vdash\conj_i) \\\\
\PGvdash\lrbbar{\overline{\conj_i\mapsto u_i}}
:\Partial{\left(\bigvee\nolimits_{i} \conj_i\right)}A}
{\PGvdash\isTypeBox{\extTy{\overline{x}}{A}{\overline{\conj_i\mapsto u_i}}}}
\and
\inferrule{\Phi\overline{,x:\II};\Gamma,\vdash v:A\\\\
\forall i. (\Phi\overline{,x:\II};\Gamma,\conj_i\vdash v\equiv u_i:A)}
{\PGvdash \plam{\overline x}{v}:\extTy{\overline{x}}{A}{\overline{\conj_i\mapsto u_i}}}
\and
\inferrule{\PGvdash \overline {u_i:\II}\\\\
\PGvdash v:\extTy{\overline{x_i}}{A}{\overline\face}}
{\PGvdash \papp v {\overline{u_i}} : A[\overline{u_i/x_i}]}
\and
\inferrule{\PGvdash \overline {u_i:\II}\\
\Phi\overline{,x_i:\II};\Gamma,\vdash v:A}
{\PGvdash \papp{(\plam{\overline{x_i}}v)}{\overline{u_i}}\equiv
v[\overline{u_i/x_i}] : A[\overline{u_i/x_i}]}
\and
\inferrule{\PGvdash \overline {u_i:\II}\\
\PGvdash v:\extTy{\overline{x_i}}{A}{\overline\face}\\\\
\PGvdash \lrbbar{\overline\face}[\overline{u_i/x_i}]\equiv v_0 : A[\overline{u_i/x_i}]}
{\PGvdash \papp v {\overline{u_i}}\equiv v_0 : A[\overline{u_i/x_i}]}
\end{mathpar}
\caption{Typing rules of generalized paths}\label{fig:tyck-path}
\end{figure}

\begin{example}\label{ex:gconcat}
The path concatenation operation can have a simpler type signature.
With the path type, its type is:
\[\textsf{concat}~(a:A)~(b:A)~(c:A)~(p:\PathTy{\lam \_ A} a b)
~(q:\PathTy{\lam \_ A} b c):\PathTy{\lam \_ A} a c\]
The first three parameters can be inferred from the last two,
but we have to quantify over them anyway.
Using the generalized path type, we may simplify it as:
\begin{align*}
\textsf{concat}~&(p:\extTy{\_}{A}{})~(q:\extTy{\_}{A}{x=\lcon\mapsto p~@~\rcon})\\
&:{\ExtTy{\_}{A}{rll}{x&=\lcon&\mapsto p~@~\lcon\\
x&=\rcon&\mapsto q~@~\rcon}}
\end{align*}
It becomes longer, but the new definition has fewer parameters.
The new definition can be more friendly to program synthesizers
or programming languages that does not support implicit arguments.
\end{example}

\begin{example}\label{ex:gsquare}
Consider \fbox{$p_i:\PathTy{\lam\_ A} \bullet \bullet$}
for $i\in\set{1,2,3,4}$ and the following square \fbox{$x:\II,y:\II\vdash u$}:
\carloCTikZ{\carloXy
\carloSqBullets
\fill [pattern color=gray,pattern=north west lines] (0,0) rectangle (1,1) ;
\node (center) at (0.5, 0.5) {$u$} ;}
\end{example}
