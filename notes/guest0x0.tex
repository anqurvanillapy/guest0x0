\section{Introduction}
\subsection{Target Audience}
This tutorial assumes familiarity with the following:
\begin{itemize}
\item Dependent type theory concepts, such as formation rules,
introduction rules, eliminators, etc., and functional programming.
\item Programming and proving in a proof assistant based on dependent type theories.
\item The ability to translate (simple) typing rules into a type-checking algorithm.
\item Basic understanding of De Morgan cubical type theory~\cite{CCHM,CHM},
including the interval type, the path type,
and the idea of representing cubes using terms with interval variables in it.
\end{itemize}
For the last point, understanding the path concatenation in cubical type theory
should be sufficient.
\subsection{Motivation}
This document is intended to help readers get more familiar with how
cubical type theory works, what it can do and what it cannot do.
Cubical type theory extends Martin-L\"{o}f type theory with huge new constructions,
especially the typing rules are written in a very compact way.

