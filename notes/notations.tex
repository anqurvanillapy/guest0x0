\newcommand{\GuestName}{\textsc{Guest0x0}}
\newcommand{\CTT}{{\mancube}TT}
\newcommand{\II}{\mathbb{I}}
\newcommand{\hcomp}{\textsf{hcomp}}
\newcommand{\concat}{\textsf{concat}}
\newcommand{\sym}{\textsf{sym}}
\newcommand{\PGvdash}{\Phi;\Gamma\vdash}
\newcommand{\lcon}{\textsf{left}}
\newcommand{\rcon}{\textsf{right}}
\newcommand{\cond}{\textsf{cond}}
\newcommand{\face}{\textsf{face}}
\newcommand{\Partial}[2]{\textsf{Partial}~{#1}~{#2}}
\newcommand{\conj}{{\phi^\land}}
\newcommand{\disj}{{\phi^\lor}}
\newcommand{\Path}[3]{\textsf{Path}~{#1}~{#2}~{#3}}

% carlo: https://gist.github.com/cangiuli/b21c9f8cb49dde06eef6480c29f7cf21
\pgfdeclarelayer{frontmost}
\pgfsetlayers{main,frontmost}
\usetikzlibrary{patterns}

\tikzset{
  carlo-axes/.style =
  {
    y = {(0,-1)},
    z = {(-0.6,0.6)}
  } ,
  shorten <>/.style =
  {
    shorten >=#1 , shorten <=#1
  } ,
  equals arrow/.style =
  {
    arrows = - ,
    double equal sign distance ,
  } ,
}

\newcommand{\carloCubeBullets}{\begin{pgfonlayer}{frontmost}
  \foreach \x in {0,1}
    \foreach \y in {0,1}
      \foreach \z in {0,1}
        \node (\x\y\z) at (\x , \y , \z) {\textbullet} ;
\end{pgfonlayer}}
\newcommand{\carloSqBullets}{\foreach \x in {0,1}
\foreach \y in {0,1}
  \node (\x\y) at (\x , \y) {\textbullet} ;}
\newcommand{\carloTikZ}[1]{
  \begin{tikzpicture}[carlo-axes, scale = 1.6]
		#1
	\end{tikzpicture}}
\newcommand{\carloCoord}[3]{\begin{scope}[shift={(-1.5,0)},scale=0.4]
    \draw [->] (0,0,0) to node [above,pos=1] {#1} (1,0,0) ;
    \draw [->] (0,0,0) to node [right,pos=1] {#2} (0,1,0) ;
    \draw [->] (0,0,0) to node [above,pos=1.3] {#3} (0,0,1) ;
  \end{scope}}
\newcommand{\carloCoordSq}[2]{\begin{scope}[shift={(-1.5,0)},scale=0.4]
    \draw [->] (0,0) to node [above,pos=1] {#1} (1,0) ;
    \draw [->] (0,0) to node [right,pos=1] {#2} (0,1) ;
  \end{scope}}
\newcommand{\carloXyz}{\carloCoord{$x$}{$y$}{$z$}}
\newcommand{\carloXy}{\carloCoordSq{$x$}{$y$}}
\newcommand{\refcube}[1]{\ExecuteMetaData[cubes]{#1}}
\newcommand{\carloCTikZ}[1]{\begin{center}
\carloTikZ{#1}
\end{center}}
\newcommand{\shiftTikZ}[2]{\begin{scope}[shift={(#1)}]
#2
\end{scope}}
